\chapter{绪论}
\section{荷塘月色}
这几天心里颇不宁静。今晚在院子里坐着乘凉,忽然想起日日走过的荷塘,在这满月的光里,总该另有一番样子吧。月亮渐渐地升高了,墙外马路上孩子们的欢笑,已经听不见了;妻在屋里拍着闰儿,迷迷糊糊地哼着眠歌。我悄悄地披了大衫,带上门出去。

沿着荷塘,是一条曲折的小煤屑路。这是一条幽僻的路;白天也少人走,夜晚更加寂寞。荷塘四面,长着许多树,蓊蓊郁郁的。路的一旁,是些杨柳,和一些不知道名字的树。没有月光的晚上,这路上阴森森的,有些怕人。今晚却很好,虽然月光也还是淡淡的。

路上只我一个人,背着手踱着。这一片天地好像是我的;我也像超出了平常的自己,到了另一世界里。我爱热闹,也爱冷静;爱群居,也爱独处。像今晚上,一个人在这苍茫的月下,什么都可以想,什么都可以不想,便觉是个自由的人。白天里一定要做的事,一定要说的话,现在都可不理。这是独处的妙处,我且受用这无边的荷香月色好了。

曲曲折折的荷塘上面,弥望的是田田的叶子。叶子出水很高,像亭亭的舞女的裙。层层的叶子中间,零星地点缀着些白花,有袅娜地开着的,有羞涩地打着朵儿的;正如一粒粒的明珠,又如碧天里的星星,又如刚出浴的美人。微风过处,送来缕缕清香,仿佛远处高楼上渺茫的歌声似的。这时候叶子与花也有一丝的颤动,像闪电般,霎时传过荷塘的那边去了。叶子本是肩并肩密密地挨着,这便宛然有了一道凝碧的波痕。叶子底下是脉脉的流水,遮住了,不能见一些颜色;而叶子却更见风致了。

月光如流水一般,静静地泻在这一片叶子和花上。薄薄的青雾浮起在荷塘里。叶子和花仿佛在牛乳中洗过一样;又像笼着轻纱的梦。虽然是满月,天上却有一层淡淡的云,所以不能朗照;但我以为这恰是到了好处——酣眠固不可少,小睡也别有风味的。月光是隔了树照过来的,高处丛生的灌木,落下参差的斑驳的黑影,峭楞楞如鬼一般;弯弯的杨柳的稀疏的倩影,却又像是画在荷叶上。塘中的月色并不均匀;但光与影有着和谐的旋律,如梵婀玲上奏着的名曲。

荷塘的四面,远远近近,高高低低都是树,而杨柳最多。这些树将一片荷塘重重围住;只在小路一旁,漏着几段空隙,像是特为月光留下的。树色一例是阴阴的,乍看像一团烟雾;但杨柳的丰姿,便在烟雾里也辨得出。树梢上隐隐约约的是一带远山,只有些大意罢了。树缝里也漏着一两点路灯光,没精打采的,是渴睡人的眼。这时候最热闹的,要数树上的蝉声与水里的蛙声;但热闹是它们的,我什么也没有。

忽然想起采莲的事情来了。采莲是江南的旧俗,似乎很早就有,而六朝时为盛;从诗歌里可以约略知道。采莲的是少年的女子,她们是荡着小船,唱着艳歌去的。采莲人不用说很多,还有看采莲的人。那是一个热闹的季节,也是一个风流的季节。梁元帝《采莲赋》里说得好:

于是妖童媛女,荡舟心许;鷁首徐回,兼传羽杯;欋将移而藻挂,船欲动而萍开。尔其纤腰束素,迁延顾步;夏始春余,叶嫩花初,恐沾裳而浅笑,畏倾船而敛裾。

可见当时嬉游的光景了。这真是有趣的事,可惜我们现在早已无福消受了。

于是又记起《西洲曲》里的句子:

采莲南塘秋,莲花过人头;低头弄莲子,莲子清如水。今晚若有采莲人,这儿的莲花也算得“过人头”了;只不见一些流水的影子,是不行的。这令我到底惦着江南了。——这样想着,猛一抬头,不觉已是自己的门前;轻轻地推门进去,什么声息也没有,妻已睡熟好久了。

\subsection{劝学}
君子曰:学不可以已。

青,取之于蓝而青于蓝;冰,水为之而寒于水。木直中绳,輮(左应为“车”,原字已废除)以为轮,其曲中规。虽有槁暴,不复挺者,輮使之然也。故木受绳则直,金就砺则利,君子博学而日参省乎己,则知明而行无过矣。

故不登高山,不知天之高也;不临深溪,不知地之厚也;不闻先王之遗言,不知学问之大也。干、越、夷、貉之子,生而同声,长而异俗,教使之然也。诗曰:“嗟尔君子,无恒安息。靖共尔位,好是正直。神之听之,介尔景福。”神莫大于化道,福莫长于无祸。

吾尝终日而思矣,不如须臾之所学也;吾尝跂而望矣,不如登高之博见也。登高而招,臂非加长也,而见者远;顺风而呼,声非加疾也,而闻者彰。假舆马者,非利足也,而致千里;假舟楫者,非能水也,而绝江河。君子生非异也,善假于物也。

南方有鸟焉,名曰蒙鸠,以羽为巢,而编之以发,系之苇苕,风至苕折,卵破子死。巢非不完也,所系者然也。西方有木焉,名曰射干,茎长四寸,生于高山之上,而临百仞之渊,木茎非能长也,所立者然也。蓬生麻中,不扶而直;白沙在涅,与之俱黑。兰槐之根是为芷,其渐之滫,君子不近,庶人不服。其质非不美也,所渐者然也。故君子居必择乡,游必就士,所以防邪辟而近中正也。

物类之起,必有所始。荣辱之来,必象其德。肉腐出虫,鱼枯生蠹。怠慢忘身,祸灾乃作。强自取柱,柔自取束。邪秽在身,怨之所构。施薪若一,火就燥也,平地若一,水就湿也。草木畴生,禽兽群焉,物各从其类也。是故质的张,而弓矢至焉;林木茂,而斧斤至焉;树成荫,而众鸟息焉。醯酸,而蚋聚焉。故言有招祸也,行有招辱也,君子慎其所立乎!

积土成山,风雨兴焉;积水成渊,蛟龙生焉;积善成德,而神明自得,圣心备焉。故不积跬步,无以至千里;不积小流,无以成江海。骐骥一跃,不能十步;驽马十驾,功在不舍。锲而舍之,朽木不折;锲而不舍,金石可镂。蚓无爪牙之利,筋骨之强,上食埃土,下饮黄泉,用心一也。蟹六跪而二螯,非蛇鳝之穴无可寄托者,用心躁也。

是故无冥冥之志者,无昭昭之明;无惛惛之事者,无赫赫之功。行衢道者不至,事两君者不容。目不能两视而明,耳不能两听而聪。螣蛇无足而飞,鼫鼠五技而穷。《诗》曰:“尸鸠在桑,其子七兮。淑人君子,其仪一兮。其仪一兮,心如结兮!”故君子结于一也。

昔者瓠巴鼓瑟,而流鱼出听;伯牙鼓琴,而六马仰秣。故声无小而不闻,行无隐而不形 。玉在山而草润,渊生珠而崖不枯。为善不积邪?安有不闻者乎?

学恶乎始?恶乎终?曰:其数则始乎诵经,终乎读礼;其义则始乎为士,终乎为圣人,真积力久则入,学至乎没而后止也。故学数有终,若其义则不可须臾舍也。为之,人也;舍之,禽兽也。故书者,政事之纪也;诗者,中声之所止也;礼者,法之大分,类之纲纪也。故学至乎礼而止矣。夫是之谓道德之极。礼之敬文也,乐之中和也,诗书之博也,春秋之微也,在天地之间者毕矣。君子之学也,入乎耳,着乎心,布乎四体,形乎动静。端而言,蝡而动,一可以为法则。小人之学也,入乎耳,出乎口;口耳之间,则四寸耳,曷足以美七尺之躯哉!古之学者为己,今之学者为人。君子之学也,以美其身;小人之学也,以为禽犊。故不问而告谓之傲,问一而告二谓之囋。傲、非也,囋、非也;君子如向矣。
 
学莫便乎近其人。礼乐法而不说,诗书故而不切,春秋约而不速。方其人之习君子之说,则尊以遍矣,周于世矣。故曰:学莫便乎近其人。

学之经莫速乎好其人,隆礼次之。上不能好其人,下不能隆礼,安特将学杂识志,顺诗书而已耳。则末世穷年,不免为陋儒而已。将原先王,本仁义,则礼正其经纬蹊径也。若挈裘领,诎五指而顿之,顺者不可胜数也。不道礼宪,以诗书为之,譬之犹以指测河也,以戈舂黍也,以锥餐壶也,不可以得之矣。故隆礼,虽未明,法士也;不隆礼,虽察辩,散儒也。
问楛者,勿告也;告楛者,勿问也;说楛者,勿听也。有争气者,勿与辩也。故必由其道至,然后接之;非其道则避之。故礼恭,而后可与言道之方;辞顺,而后可与言道之理;色从而后可与言道之致。故未可与言而言,谓之傲;可与言而不言,谓之隐;不观气色而言,谓瞽。故君子不傲、不隐、不瞽,谨顺其身。诗曰:“匪交匪舒,天子所予。”此之谓也。

百发失一,不足谓善射;千里蹞步不至,不足谓善御;伦类不通,仁义不一,不足谓善学。学也者,固学一之也。一出焉,一入焉,涂巷之人也;其善者少,不善者多,桀纣盗跖也;全之尽之,然后学者也。

君子知夫不全不粹之不足以为美也,故诵数以贯之,思索以通之,为其人以处之,除其害者以持养之。使目非是无欲见也,使耳非是无欲闻也,使口非是无欲言也,使心非是无欲虑也。及至其致好之也,目好之五色,耳好之五声,口好之五味,心利之有天下。是故权利不能倾也,群众不能移也,天下不能荡也。生乎由是,死乎由是,夫是之谓德操。德操然后能定,能定然后能应。能定能应,夫是之谓成人。天见其明,地见其光,君子贵其全也。